\section{Introdução}
No escopo de disciplina de Métodos Formais o segundo trabalho pode ser
desenvolvido da seguinte maneira: dado dois algoritmos, na forma de pseudo
código, deve ser demonstrada a prova de correção utilizando os conceitos da
lógica de Hoare.

A enunciado da tarefa apresenta dois algoritmos no seguinte formato:

\begin{minipage}{6cm}
\null 
  \begin{algorithm}[H]
    \caption{Algoritmo 1}
      \Function{Algoritmo 01}{$x$}
      \State $i\gets 1$
      \State $j\gets 4$
      \While{$i\ \ne \ x$}
      \State $j \gets j\ +\ 2\ *\ i\ +\ 3$
      \State $i\gets i\ +\ 1$
      \EndWhile\label{euclidendwhile}
      \EndFunction
  \end{algorithm}
\end{minipage}%
\begin{minipage}{6cm}
\null
  \begin{algorithm}[H]
    \caption{Algoritmo 2}
      \Function{Algoritmo 02}{$x,\ y,\ n$}
      \State $i\gets 0$
      \State $j\gets x$
      \While{$i\ \ne \ n$}
      \State $j \gets j\ *\ i$
      \State $i\gets i\ +\ 1$
      \EndWhile\label{euclidendwhile}
      \EndFunction
  \end{algorithm}
\end{minipage}

\vspace{0.3cm}

Após a definiçãos dos algoritmos é necessário explicar com os devidos detalhes a
função que cada um computa, bem como a tupla de Hoare correspondente. Também
será definido, para cada um dos algoritmos, a invariante do laço e sua prova por
indução correspondente. A seguir será abordada a prova de correção parcial para
cada tupla de Hoare que for apresentada.
