\section{Algoritmo 1}
\label{sec:alg1}

O Algortimo 1 tem o objetivo de computar a seguinte operação: dado uma variável
\emph{x} o algoritmo retorna $(x\ +\ 1)^2$ e pode ser representado com a
seguinte tupla de hoare:

$$(\!|\ x > 0\ |\!)\ i := 1;\ j := 4\ \text{while}\ i \neq x \left\{j := j + 2 \times i + 3;\ i := i + 1\right\}\ (\!|\ j = (x + 1)^2\ |\!)$$

Depois de termos provado a invariante do \emph{laço} \[INV = j = (i + 1)^2
\land i \leq x\] estamos aptos para verificar se o programa correspondente ao
algoritmo 1 está correto.

\begin{equation*}
  \begin{fitch}
    \fa (\!|\ j = (i + 2)^2 \land i + 1 \leq x\ |\!)\ i := i + 1\ (\!|\ INV\ |\!) & ATR \\
    \fa 
      \begin{split}
        &(\!|\ j + 2 \times i + 3 = (i + 2)^2 \land i + 1 \leq x\ |\!) \\
        &j := j + 2 \times i + 3 \\
        &(\!|\ j = (i + 2)^2 \land i + 1 \leq x\ |\!) \\
      \end{split}
      & ATR \\
    \fa 
      \begin{split}
        &(\!|\ j + 2 \times i + 3 = (i + 2)^2 \land i + 1 \leq x\ |\!) \\
        &j := j + 2 \times i + 3;\ i := i + 1 \\
        &(\!|\ INV\ |\!) \\ 
      \end{split}
      & 1,2COMP \\
    \fa \fh (j = (i + 1)^2 \land i \leq x) \land i \neq x & H \\
    \fa \fa j = (i + 1)^2 \land i \leq x & 4, $\land$EL \\
    \fa \fa j = (i + 1)^2 & 5, $\land$EL \\
    \fa \fa i \leq x & 5, $\land$ER \\
    \fa \fa j = i^2 + 2 \times i + 1 & 6, ARIT \\
    \fa \fa j + 3 = i^2 + 2 \times i + 4 & 8, ARIT \\
    \fa \fa j + 2 \times i + 3 = i^2 + 4i + 4 & 9, ARIT \\
    \fa \fa j + 2 \times i + 3 = (i + 2)^2 & 10, ARIT \\
    \fa \fa i < x \lor i = x & 7, DEF$\leq$ \\
    \fa \fa i \neq x & 4, $\land$ER \\
    \fa \fa (i < x \lor i = x) \land i \neq x & 12,13 $\land$I \\
    \fa \fa i < x \land i \neq x \lor i = x \land i \neq x & 14, DIST$\land$ \\
    \fa \fa i < x \land i \neq x \lor \bot & 15, CONTRADIÇÃO \\
    \fa \fa i < x \land i \neq x & 16, LÓGICA \\
    \fa \fa i < x & 17, $\land$EL \\
    \fa \fa i + 1 \leq x & 18, ARIT \\
    \fa \fa j + 2 \times i + 3 = (i + 2)^2 \land i + 1 \leq x & 11,19 $\land$I \\
    \fa
      \begin{split}
        &(j = (i + 2)^2 \land i \leq x) \land i \neq x \\
        &\rightarrow \\
        &j + 2 \times i + 3 = (i + 2)^2 \land i + 1 \leq x \\
      \end{split}
      & 4-20, $\rightarrow$I \\
    \fa 
      \begin{split}
        &(\!|\ (j = (i + 1)^2 \land i \leq x) \land i \neq x\ |\!) \\
        &j := j + 2 \times i + 3;\ i := i + 1 \\
        &(\!|\ INV\ |\!) \\
      \end{split}
      & 3,21,PreStren \\
    \fa 
      \begin{split}
        &(\!|\ INV\ |\!) \\
        &\text{while}\ i \neq x \left\{j := j + 2 \times i + 3;\ i := i + 1\right\} \\
        &(\!|\ INV \land \neg (i \neq x)\ |\!) \\
      \end{split}
      & 22, PWhile \\
  \end{fitch}
\end{equation*}

\newpage

Agora vemos que:

\begin{equation*}
  \begin{fitch}
    \fa (\!|\ 4 = (i + 1)^2 \land i \leq x\ |\!)\ j := 4\ (\!|\ INV\ |\!) & ATR \\
    \fa (\!|\ 4 = (1 + 1)^2 \land 1 \leq x\ |\!)\ i := 1\ (\!|\ 4 = (i + 1)^2 \land i \leq x\ |\!) & ATR \\
    \fa (\!|\ 4 = (1 + 1)^2 \land 1 \leq x\ |\!)\ i := 1;\ j := 4\ (\!|\ INV\ |\!) & 1,2,COMP \\
    \fa \fh x > 0 & H \\
    \fa \fa 4 = 4 & = \\
    \fa \fa 4 = 2^2 & 5, ARIT \\
    \fa \fa 4 = (1 + 1)^2 & 6, ARIT \\
    \fa \fa x \geq 1 & 4, ARIT \\
    \fa \fa 1 \leq x & 8, ARIT \\
    \fa \fa 4 = (1 + 1)^2 \land 1 \leq x & 7,9,$\land$I \\
    \fa x > 0 \rightarrow 4 = (1 + 1)^2 \land 1 \leq x & 4-10, $\rightarrow$I \\
    \fa (\!|\ x > 0\ |\!)\ i := 1;\ j := 4\ (\!|\ INV\ |\!) & 3,11, PreStren \\
  \end{fitch}
\end{equation*}

Com o \emph{while}, que foi concluído na primeira parte, e a composição, que foi
concluída na seguinda parte, podemos seguir a prova desta maneira:

\begin{equation*}
  \begin{fitch}
    \fb
      \begin{split}
        &(\!|\ INV\ |\!) \\
        &\text{while}\ i \neq x \left\{j := j + 2 \times i + 3;\ i := i + 1\right\} \\
        &(\!|\ INV \land \neg (i \neq x)\ |\!) \\
      \end{split}
      & H \\
    \fj (\!|\ x > 0\ |\!)\ i := 1;\ j := 4\ (\!|\ INV\ |\!) & H \\
    \fa 
      \begin{split}
        &(\!|\ x > 0\ |\!) \\
        &i := 1;\ j := 4\ \text{while}\ i \neq x \left\{j := j + 2 \times i + 3;\ i := i + 1\right\} \\
        &(\!|\ INV \land \neg (i \neq x)\ |\!) \\
      \end{split}
      & 2,1, COMP \\
    \fa \fh (j = (i + 1)^2 \land i \leq x) \land \neg (i \neq x) & H \\
    \fa \fa j = (i + 1)^2 \land i \leq x & 4, $\land$EL \\
    \fa \fa j = (i + 1)^2 & 5, $\land$EL \\
    \fa \fa \neg (i \neq x) & 4, $\land$ER \\
    \fa \fa i = x & 7, LÓGICA \\
    \fa \fa j = (x + 1)^2 & =E$\left\{6,8\right\}$ \\
    \fa INV \land \neg (i \neq x) \rightarrow j = (x + 1)^2 & 4-9, $\rightarrow$I \\
    \fa
      \begin{split}
        &(\!|\ x > 0\ |\!) \\
        &i := 1;\ j := 4\ \text{while}\ i \neq x \left\{j := j + 2 \times i + 3;\ i := i + 1\right\} \\
        &(\!|\ j = (x + 1)^2\ |\!) \\
      \end{split}
      & 3,10,PosWeak \\
  \end{fitch}
\end{equation*}
