\section{Algoritmo 1}
\label{sec:alg1}

O Algortimo 1 tem o objetivo de computar a seguinte operação: dado uma variável
\emph{x} o algoritmo retorna $(x\ +\ 1)^2$ e pode ser representado com a
seguinte tupla de hoare:

$$(\!|\ x > 0\ |\!)\ i := 1;\ j := 4\ \text{while}\ i \neq x \left\{j := j + 2 \times i + 3;\ i := i + 1\right\}\ (\!|\ j = (x + 1)^2\ |\!)$$

\subsubsection{Prova da Inavariante do Loop}
\label{ssub:Prova da Inavariante do Loop}

Nesta seção demonstraremos a prova da invariante do loop. Após ser feita a
análise do comportamento das variáveis duranda a execução do programa chegamos a
conclusão que a invariate do loop era:

$$ j\  =\ (i\ +\ 1)^2\ \land i\ \leq\ x $$

Porém é preciso provar que a invariante é válida para qualquer momento do
código. A prova será feita utilizando indução, com isso apresentaremos a prova
para um caso base, neste caso a base será repsentada pelo número zero que marca
que o laço não ocorreu. Assumiremos a validade da invariante para \emph{n} e
desmontraremos que a também se mantém vardadeira para \emph{n + 1}.

\paragraph{Prova:} 
Devemos provar o seguinte: $P \triangleq \forall_k :\mathbb{N} .\ j_k=(i_k\ +\ 1)^2\ \land\
i_k \leq x$. Assumindo o teorema da distributividade do para todo temos o
seguinte:

$$\forall_w :\mathbb{N}.\ j_w=(i_w\ +1)^2\ \land  \forall_z:\mathbb{N}. i_z \leq x$$

Com isso devemos demonstrar a prova para cada lado da conjunção. Agora Veja:

\paragraph{Caso Base P(0)}
É necessário provar o seguinte: $ j_0 =(i_0\ +\ 1)^2 $ Agora veja que:
 \begin{center}
     \begin{tabular}{rll}
         \emph{$j_0$} =& \emph{4} & \emph{(j := 4)} \\
                      =& \emph{$2^2$}& \emph{(Artmimética)}\\ 
                      =& \emph{$(1\ +\ 1)^2$}& \emph{(Aritmética)}\\        
                      =& \emph{$(i_0\ +\ 1)^2$}& \emph{(i := 1)}\\ 
         ~&~&~\\
         ~&~&q.e.d\\
     \end{tabular}
 \end{center}
Também precisamos provar o seguinte:$i_0 \leq x$ Agora veja que:
 \begin{center}
     \begin{tabular}{rl}
         \emph{$x\ \geq\ 0$}& \emph{PRE} \\
         \emph{$x\ \geq\ 1$}& \emph{Aritimética} \\
         \emph{$x\ \geq\ i_0$}& \emph{$i_0$ := 1} \\
         \emph{$i_0\ \leq\ x$}& \emph{Aritimética} \\
         ~&~\\
         ~&q.e.d\\
     \end{tabular}
 \end{center}

\paragraph{Caso Indutivo P(n)}
Assumindo \emph{n} como um valor arbitrário e como hipótese de indução
$\forall k :\mathbb{N} .\ j_k=(i_k\ +\ 1)^2$ temos que mostrar que: 
$j_{k+1}=(i_{k+1}\ +\ 1)^2$ agora veja que:
 \begin{center}
     \begin{tabular}{rll}
         \emph{$(i_{k+1}\ +\ 1)$} =& \emph{$((i_w\ +\ 1)\ +\ 1)^2$} & \emph{(i := i + 1 )} \\
                                  =& \emph{$(i_w + 1)^2 + 2(i_w +1) + 1$}& \emph{(Artmimética)}\\ 
                                  =& \emph{$j_w + 2(i_w + 1) + 1$}& \emph{(H.I)}\\        
                                  =& \emph{$j_w + 2i_w + 3$}& \emph{(Aritimética)}\\ 
         =& \emph{$j_{w+1} $}& \emph{(j := j + 2*i + 3)}\\ 
         ~&~&~\\
         ~&~&q.e.d\\
     \end{tabular}
 \end{center}

 Também precisamos provar assumindo \emph{z} como um valor arbitrário e como
 hipótese de indução $\forall z : \mathbb{N}\ .\ i_z \leq x$ temos que mostrar
 que: $i_{z+1} \leq x$ agora veja que:
 \begin{center}
     \begin{tabular}{rl}
         \emph{$i_z \leq x \land \neg(i_z = x) $}& \emph{$\land I$, HI, LOOP} \\
         \emph{$\neg(i_z=x) \land (i_z =x) \vee \neg(i_z=z) \land (i_z < x)$}&\emph{Distributividade Conjunção} \\
         \emph{$\neg(i_z =x) \land (i_z < x)$}& \emph{Lógica} \\
         \emph{$i_z < x$}& \emph{$\land \varepsilon$} \\
         \emph{$i_z +1 \leq x$}& \emph{Aritimética} \\
         \emph{$i_{z+1} \leq x$}& \emph{Aritimética} \\
         ~&~\\
         ~&q.e.d\\
     \end{tabular}
 \end{center}
\subsection{Prova da Tripla de Hoare}
\label{sub:hoarepow}

Depois de termos provado a invariante do \emph{laço} \[INV = j = (i + 1)^2
\land i \leq x\] estamos aptos para verificar se o programa correspondente ao
Algoritmo 1 está correto.

\begin{equation*}
  \begin{fitch}
    \fa (\!|\ j = (i + 2)^2 \land i + 1 \leq x\ |\!)\ i := i + 1\ (\!|\ INV\ |\!) & ATR \\
    \fa 
      \begin{split}
        &(\!|\ j + 2 \times i + 3 = (i + 2)^2 \land i + 1 \leq x\ |\!) \\
        &j := j + 2 \times i + 3 \\
        &(\!|\ j = (i + 2)^2 \land i + 1 \leq x\ |\!) \\
      \end{split}
      & ATR \\
    \fa 
      \begin{split}
        &(\!|\ j + 2 \times i + 3 = (i + 2)^2 \land i + 1 \leq x\ |\!) \\
        &j := j + 2 \times i + 3;\ i := i + 1 \\
        &(\!|\ INV\ |\!) \\ 
      \end{split}
      & 1,2COMP \\
    \fa \fh (j = (i + 1)^2 \land i \leq x) \land i \neq x & H \\
    \fa \fa j = (i + 1)^2 \land i \leq x & 4, $\land$EL \\
    \fa \fa j = (i + 1)^2 & 5, $\land$EL \\
    \fa \fa i \leq x & 5, $\land$ER \\
    \fa \fa j = i^2 + 2 \times i + 1 & 6, ARIT \\
    \fa \fa j + 3 = i^2 + 2 \times i + 4 & 8, ARIT \\
    \fa \fa j + 2 \times i + 3 = i^2 + 4i + 4 & 9, ARIT \\
    \fa \fa j + 2 \times i + 3 = (i + 2)^2 & 10, ARIT \\
    \fa \fa i < x \lor i = x & 7, DEF$\leq$ \\
    \fa \fa i \neq x & 4, $\land$ER \\
    \fa \fa (i < x \lor i = x) \land i \neq x & 12,13 $\land$I \\
    \fa \fa i < x \land i \neq x \lor i = x \land i \neq x & 14, DIST$\land$ \\
    \fa \fa i < x \land i \neq x \lor \bot & 15, CONTRADIÇÃO \\
    \fa \fa i < x \land i \neq x & 16, LÓGICA \\
    \fa \fa i < x & 17, $\land$EL \\
    \fa \fa i + 1 \leq x & 18, ARIT \\
    \fa \fa j + 2 \times i + 3 = (i + 2)^2 \land i + 1 \leq x & 11,19 $\land$I \\
    \fa
      \begin{split}
        &(j = (i + 2)^2 \land i \leq x) \land i \neq x \\
        &\rightarrow \\
        &j + 2 \times i + 3 = (i + 2)^2 \land i + 1 \leq x \\
      \end{split}
      & 4-20, $\rightarrow$I \\
    \fa 
      \begin{split}
        &(\!|\ (j = (i + 1)^2 \land i \leq x) \land i \neq x\ |\!) \\
        &j := j + 2 \times i + 3;\ i := i + 1 \\
        &(\!|\ INV\ |\!) \\
      \end{split}
      & 3,21,PreStren \\
    \fa 
      \begin{split}
        &(\!|\ INV\ |\!) \\
        &\text{while}\ i \neq x \left\{j := j + 2 \times i + 3;\ i := i + 1\right\} \\
        &(\!|\ INV \land \neg (i \neq x)\ |\!) \\
      \end{split}
      & 22, PWhile \\
  \end{fitch}
\end{equation*}

\newpage

Agora vemos que:

\begin{equation*}
  \begin{fitch}
    \fa (\!|\ 4 = (i + 1)^2 \land i \leq x\ |\!)\ j := 4\ (\!|\ INV\ |\!) & ATR \\
    \fa (\!|\ 4 = (1 + 1)^2 \land 1 \leq x\ |\!)\ i := 1\ (\!|\ 4 = (i + 1)^2 \land i \leq x\ |\!) & ATR \\
    \fa (\!|\ 4 = (1 + 1)^2 \land 1 \leq x\ |\!)\ i := 1;\ j := 4\ (\!|\ INV\ |\!) & 1,2,COMP \\
    \fa \fh x > 0 & H \\
    \fa \fa 4 = 4 & = \\
    \fa \fa 4 = 2^2 & 5, ARIT \\
    \fa \fa 4 = (1 + 1)^2 & 6, ARIT \\
    \fa \fa x \geq 1 & 4, ARIT \\
    \fa \fa 1 \leq x & 8, ARIT \\
    \fa \fa 4 = (1 + 1)^2 \land 1 \leq x & 7,9,$\land$I \\
    \fa x > 0 \rightarrow 4 = (1 + 1)^2 \land 1 \leq x & 4-10, $\rightarrow$I \\
    \fa (\!|\ x > 0\ |\!)\ i := 1;\ j := 4\ (\!|\ INV\ |\!) & 3,11, PreStren \\
  \end{fitch}
\end{equation*}

Com o \emph{while}, que foi concluído na primeira parte, e a composição, que foi
concluída na seguinda parte, podemos seguir a prova desta maneira:

\begin{equation*}
  \begin{fitch}
    \fb
      \begin{split}
        &(\!|\ INV\ |\!) \\
        &\text{while}\ i \neq x \left\{j := j + 2 \times i + 3;\ i := i + 1\right\} \\
        &(\!|\ INV \land \neg (i \neq x)\ |\!) \\
      \end{split}
      & \\
    \fj (\!|\ x > 0\ |\!)\ i := 1;\ j := 4\ (\!|\ INV\ |\!) & \\
    \fa 
      \begin{split}
        &(\!|\ x > 0\ |\!) \\
        &i := 1;\ j := 4\ \text{while}\ i \neq x \left\{j := j + 2 \times i + 3;\ i := i + 1\right\} \\
        &(\!|\ INV \land \neg (i \neq x)\ |\!) \\
      \end{split}
      & 2,1, COMP \\
    \fa \fh (j = (i + 1)^2 \land i \leq x) \land \neg (i \neq x) & H \\
    \fa \fa j = (i + 1)^2 \land i \leq x & 4, $\land$EL \\
    \fa \fa j = (i + 1)^2 & 5, $\land$EL \\
    \fa \fa \neg (i \neq x) & 4, $\land$ER \\
    \fa \fa i = x & 7, LÓGICA \\
    \fa \fa j = (x + 1)^2 & =E$\left\{6,8\right\}$ \\
    \fa INV \land \neg (i \neq x) \rightarrow j = (x + 1)^2 & 4-9, $\rightarrow$I \\
    \fa
      \begin{split}
        &(\!|\ x > 0\ |\!) \\
        &i := 1;\ j := 4\ \text{while}\ i \neq x \left\{j := j + 2 \times i + 3;\ i := i + 1\right\} \\
        &(\!|\ j = (x + 1)^2\ |\!) \\
      \end{split}
      & 3,10,PosWeak \\
  \end{fitch}
\end{equation*}
