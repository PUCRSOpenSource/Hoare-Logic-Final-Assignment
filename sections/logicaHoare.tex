\section{Lógica de Hoare}
\label{sec:LógicadeHoare}

Lógica de Hoare é um sistema formal com um conjunto de regras lógicas para um
raciocínio rigoroso sobre a corretude de um programa de computador.

\subsection{Tripla de Hoare}
\label{sub:tripla}

Uma das mais importantes propriedades de um programa é se ele faz ou não faz a
sua função. A intencional função de um programa, ou parte de um programa, pode
ser especificada fazendo asserções gerais sobre os valores em que as variáveis
relevantes vão ter \emph{depois} da execução do programa. Essas asserções
geralmente não vão atribuir valores particulares para cada variável, mas sim
especificar certas propriedades gerais dos valores e das relações entre elas.

Em muitos casos, a validade dos resultados de um programa, ou parte de um
programa, vai depender dos valores tomados pelas variáveis antes desse programa
ser iniciado. Essas precondições iniciais de uso bem sucedido podem ser
especificadas pelo mesmo tipo de asserção geral como é usado para descrever os
resultados obtidos na terminação. Para estabelecer a conexão requirida entre a
precondição ($P$), um programa ($Q$) e a descrição do resultado da sua execução
($R$), foi introduzido uma nova notação: \[(\!|\  P\  |\!)\; Q\; (\!|\  R\
|\!)\]

Isso pode ser interpretado como "Se a asserção $P$ for verdadeira antes do
inícido do programa $Q$, então a asserção $R$ será verdadeira após sua
execução." Se não houver precondições impostas, escrevemos como  $(\!|\  true\
|\!)\; Q\; (\!|\  R\ |\!)$\cite{hoare:68}.

