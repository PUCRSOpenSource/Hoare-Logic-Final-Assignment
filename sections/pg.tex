\newpage

\section{Algoritmo 2}
\label{sec:algo2}

O algotimo 2 retorna o n-ésimo termo de uma progressão geométrica de razão
\emph{y} e como termo inicial o valor \emph{x}. Este algoritmo pode ser
expresso, utilizando a notação de Hoare, da seguinte forma:

\subsubsection{Prova da Inavariante do Loop}

Nesta seção demonstraremos a prova da invariante do loop. Após ser feita a
análise do comportamento das variáveis duranda a execução do programa chegamos a
conclusão que a invariate do loop era:

$$ j\  =\ xy^{i_0}\ \land n\ \geq\ i $$

Porém é preciso provar que a invariante é válida para qualquer momento do
código. A prova será feita utilizando indução, com isso apresentaremos a prova
para um caso base, neste caso a base será repsentada pelo número zero que marca
que o laço não ocorreu. Assumiremos a validade da invariante para \emph{n} e
desmontraremos que a também se mantém vardadeira para \emph{n + 1}.

\paragraph{Prova:} 
Devemos provar o seguinte: $P \triangleq \forall_k :\mathbb{N} .\ j_k=xy^{i_n}\ \land\
n \geq i_n$. Assumindo o teorema da distributividade do para todo temos o
seguinte:

$$\forall_w :\mathbb{N}.\ j_w=xy^{i_w}\ \land  \forall_z:\mathbb{N}. n \geq i_z$$

Com isso devemos demonstrar a prova para cada lado da conjunção. Agora Veja:

\paragraph{Caso Base P(0)}
É necessário provar o seguinte: $ j_0 =xy^{i_0} $ Agora veja que:
 \begin{center}
     \begin{tabular}{rll}
         \emph{$j_0$} =& \emph{x} & \emph{(j := x)} \\
                      =& \emph{$x*1$}& \emph{(Artmimética)}\\ 
                      =& \emph{$x*y^0$}& \emph{(Aritimética)}\\        
                      =& \emph{$xy^{i_0}$}& \emph{(i := 0)}\\ 
         ~&~&~\\
         ~&~&q.e.d\\
     \end{tabular}
 \end{center}
Também precisamos provar o seguinte:$n \geq i_0$ Agora veja que:
 \begin{center}
     \begin{tabular}{rl}
         \emph{$i_0\ =\ 0$}& \emph{i:=0} \\
         \emph{$n\ \geq\ 0$}& \emph{PRE} \\
         \emph{$n\ \geq\ i_0$}& \emph{Aritimética} \\
         ~&~\\
         ~&q.e.d\\
     \end{tabular}
 \end{center}

\paragraph{Caso Indutivo P(n)}
Assumindo \emph{w} como um valor arbitrário e como hipótese de indução
$\forall w :\mathbb{N} .\ j_w=xy^i_w$ temos que mostrar que: 
$j_{w+1}=xy^{i_{w+1}}$ agora veja que:
 \begin{center}
     \begin{tabular}{rll}
         \emph{$xy^{i_w+1}$} =& \emph{$(xy^w)y$} & \emph{(Aritimética)} \\
                                  =& \emph{$j_w*y$}& \emph{(Artmimética)}\\ 
                                  =& \emph{$j_{w+1}$}& \emph{(j := j*y)}\\        
         ~&~&~\\
         ~&~&q.e.d\\
     \end{tabular}
 \end{center}

 Também precisamos provar assumindo \emph{z} como um valor arbitrário e como
 hipótese de indução $\forall z : \mathbb{N}\ .\ i_z \leq x$ temos que mostrar
 que: $i_{z+1} \leq x$ agora veja que:
 \begin{center}
     \begin{tabular}{rl}
         \emph{$i_z \leq x \land \neg(i_z = x) $}& \emph{$\land I$, HI, LOOP} \\
         \emph{$\neg(i_z=x) \land (i_z =x) \vee \neg(i_z=z) \land (i_z < x)$}&\emph{Distributividade Conjunção} \\
         \emph{$\neg(i_z =x) \land (i_z < x)$}& \emph{Lógica} \\
         \emph{$i_z < x$}& \emph{$\land \varepsilon$} \\
         \emph{$i_z +1 \leq x$}& \emph{Aritimética} \\
         \emph{$i_{z+1} \leq x$}& \emph{Aritimética} \\
         ~&~\\
         ~&q.e.d\\
     \end{tabular}
 \end{center}
$$(\!|\ x\ \geq\ 0\ \wedge x\ =\ x_0\ \wedge\ y=y_0|\!) i\ :=\ 0; j :=\ x;\
while\ i \neq\ n \{j\ :=\ j\ *\ y;\ i\ :=\ i\ +\ 1\}(\!| j=x y^n |\!) $$
