\documentclass[10pt]{article}

\usepackage[T1]{fontenc} 
\usepackage[brazilian]{babel} 
\usepackage[utf8]{inputenc} 
\usepackage{algorithm} 
\usepackage{algpseudocode}
\usepackage{amsfonts}
\usepackage{amsmath} 
\usepackage{amssymb}
\usepackage{fitch}
\usepackage{graphicx,url,float}
\usepackage{latexsym}
\usepackage{sbc-template}

\graphicspath{{images/}}


\sloppy

\title{Prova de Correção de Programas Imperativos}

\author{Diego Pinto da Jornada, Matthias Oliveira de Nunes}

\address{Faculdade de Informática -- Pontifícia Universidade Católica do Rio Grande do Sul\\
  (PUCRS)
  \email{\{diego.jornada, matthias.nunes\}@acad.pucrs.br}
}

\begin{document} 

\maketitle

\begin{resumo}
    
    Este artigo apresenta duas provas de correção de programas imperativos, a
    partir de dois algoritmos descritos na forma de pseudo código.
    
\end{resumo}
\begin{abstract}
    
    This paper presents two proofs of correctness of imperative programs, from
    two algorithms described in pseudocode form.
    
\end{abstract}

\section{Introdução}
\label{sec:intro}

No escopo de disciplina de Métodos Formais o segundo trabalho pode ser
desenvolvido da seguinte maneira: dado dois algoritmos, na forma de pseudo
código, deve ser demonstrada a prova de correção utilizando os conceitos da
lógica de Hoare.

A enunciado da tarefa apresenta dois algoritmos no seguinte formato:

\begin{minipage}{6cm}
  \begin{algorithm}[H]
    \caption{Algoritmo 1}
    \begin{algorithmic}
      \Function{Algoritmo 01}{$x$}
      \State $i\gets 1$
      \State $j\gets 4$
      \While{$i\ \ne \ x$}
      \State $j \gets j\ +\ 2\ *\ i\ +\ 3$
      \State $i\gets i\ +\ 1$
      \EndWhile
      \EndFunction
      \end{algorithmic}
  \end{algorithm}
\end{minipage}%
\begin{minipage}{7cm}
  \begin{algorithm}[H]
    \caption{Algoritmo 2}
    \begin{algorithmic}
      \Function{Algoritmo 02}{$x,\ y,\ n$}
      \State $i\gets 0$
      \State $j\gets x$
      \While{$i\ \ne \ n$}
      \State $j \gets j\ *\ i$
      \State $i\gets i\ +\ 1$
      \EndWhile\label{euclidendwhile}
      \EndFunction
      \end{algorithmic}
  \end{algorithm}
\end{minipage}

\vspace{0.3cm}

Após a definiçãos dos algoritmos é necessário explicar com os devidos detalhes a
função que cada um computa, bem como a tupla de Hoare correspondente. Também
será definido, para cada um dos algoritmos, a invariante do laço e sua prova por
indução correspondente. Na seção~\ref{sec:alg1} será abordada a prova de
correção parcial para cada tupla de Hoare que for apresentada, mas antes disso
vamos falar um pouco sobre a lógica de hoare.


\section{Lógica de Hoare}
\label{sec:LógicadeHoare}

Lógica de Hoare é um sistema formal com um conjunto de regras lógicas para um
raciocínio rigoroso sobre a corretude de um programa de computador.

\subsection{Tripla de Hoare}
\label{sub:tripla}

Uma das mais importantes propriedades de um programa é se ele faz ou não faz a
sua função. A intencional função de um programa, ou parte de um programa, pode
ser especificada fazendo asserções gerais sobre os valores em que as variáveis
relevantes vão ter \emph{depois} da execução do programa. Essas asserções
geralmente não vão atribuir valores particulares para cada variável, mas sim
especificar certas propriedades gerais dos valores e das relações entre elas.

Em muitos casos, a validade dos resultados de um programa, ou parte de um
programa, vai depender dos valores tomados pelas variáveis antes desse programa
ser iniciado. Essas precondições iniciais de uso bem sucedido podem ser
especificadas pelo mesmo tipo de asserção geral como é usado para descrever os
resultados obtidos na terminação. Para estabelecer a conexão requirida entre a
precondição ($P$), um programa ($Q$) e a descrição do resultado da sua execução
($R$), foi introduzido uma nova notação: \[(\!|\  P\  |\!)\; Q\; (\!|\  R\
|\!)\]

Isso pode ser interpretado como "Se a asserção $P$ for verdadeira antes do
inícido do programa $Q$, então a asserção $R$ será verdadeira após sua
execução." Se não houver precondições impostas, escrevemos como  $(\!|\  true\
|\!)\; Q\; (\!|\  R\ |\!)$\cite{hoare:68}.



\section{Algoritmo 1}
\label{sec:alg1}

O Algortimo 1 tem o objetivo de computar a seguinte operação: dado uma variável
\emph{x} o algoritmo retorna $(x\ +\ 1)^2$ e pode ser representado com a
seguinte tupla de hoare:

$$(\!|\ x > 0\ |\!)\ i := 1;\ j := 4\ \text{while}\ i \neq x \left\{j := j + 2 \times i + 3;\ i := i + 1\right\}\ (\!|\ j = (x + 1)^2\ |\!)$$

Depois de termos provado a invariante do \emph{laço} \[INV = j = (i + 1)^2
\land i \leq x\] estamos aptos para verificar se o programa correspondente ao
algoritmo 1 está correto.

\begin{equation*}
  \begin{fitch}
    \fa (\!|\ j = (i + 2)^2 \land i + 1 \leq x\ |\!)\ i := i + 1\ (\!|\ INV\ |\!) & ATR \\
    \fa 
      \begin{split}
        &(\!|\ j + 2 \times i + 3 = (i + 2)^2 \land i + 1 \leq x\ |\!) \\
        &j := j + 2 \times i + 3 \\
        &(\!|\ j = (i + 2)^2 \land i + 1 \leq x\ |\!) \\
      \end{split}
      & ATR \\
    \fa 
      \begin{split}
        &(\!|\ j + 2 \times i + 3 = (i + 2)^2 \land i + 1 \leq x\ |\!) \\
        &j := j + 2 \times i + 3;\ i := i + 1 \\
        &(\!|\ INV\ |\!) \\ 
      \end{split}
      & 1,2COMP \\
    \fa \fh (j = (i + 1)^2 \land i \leq x) \land i \neq x & H \\
    \fa \fa j = (i + 1)^2 \land i \leq x & 4, $\land$EL \\
    \fa \fa j = (i + 1)^2 & 5, $\land$EL \\
    \fa \fa i \leq x & 5, $\land$ER \\
    \fa \fa j = i^2 + 2 \times i + 1 & 6, ARIT \\
    \fa \fa j + 3 = i^2 + 2 \times i + 4 & 8, ARIT \\
    \fa \fa j + 2 \times i + 3 = i^2 + 4i + 4 & 9, ARIT \\
    \fa \fa j + 2 \times i + 3 = (i + 2)^2 & 10, ARIT \\
    \fa \fa i < x \lor i = x & 7, DEF$\leq$ \\
    \fa \fa i \neq x & 4, $\land$ER \\
    \fa \fa (i < x \lor i = x) \land i \neq x & 12,13 $\land$I \\
    \fa \fa i < x \land i \neq x \lor i = x \land i \neq x & 14, DIST$\land$ \\
    \fa \fa i < x \land i \neq x \lor \bot & 15, CONTRADIÇÃO \\
    \fa \fa i < x \land i \neq x & 16, LÓGICA \\
    \fa \fa i < x & 17, $\land$EL \\
    \fa \fa i + 1 \leq x & 18, ARIT \\
    \fa \fa j + 2 \times i + 3 = (i + 2)^2 \land i + 1 \leq x & 11,19 $\land$I \\
    \fa
      \begin{split}
        &(j = (i + 2)^2 \land i \leq x) \land i \neq x \\
        &\rightarrow \\
        &j + 2 \times i + 3 = (i + 2)^2 \land i + 1 \leq x \\
      \end{split}
      & 4-20, $\rightarrow$I \\
    \fa 
      \begin{split}
        &(\!|\ (j = (i + 1)^2 \land i \leq x) \land i \neq x\ |\!) \\
        &j := j + 2 \times i + 3;\ i := i + 1 \\
        &(\!|\ INV\ |\!) \\
      \end{split}
      & 3,21,PreStren \\
    \fa 
      \begin{split}
        &(\!|\ INV\ |\!) \\
        &\text{while}\ i \neq x \left\{j := j + 2 \times i + 3;\ i := i + 1\right\} \\
        &(\!|\ INV \land \neg (i \neq x)\ |\!) \\
      \end{split}
      & 22, PWhile \\
  \end{fitch}
\end{equation*}

\newpage

Agora vemos que:

\begin{equation*}
  \begin{fitch}
    \fa (\!|\ 4 = (i + 1)^2 \land i \leq x\ |\!)\ j := 4\ (\!|\ INV\ |\!) & ATR \\
    \fa (\!|\ 4 = (1 + 1)^2 \land 1 \leq x\ |\!)\ i := 1\ (\!|\ 4 = (i + 1)^2 \land i \leq x\ |\!) & ATR \\
    \fa (\!|\ 4 = (1 + 1)^2 \land 1 \leq x\ |\!)\ i := 1;\ j := 4\ (\!|\ INV\ |\!) & 1,2,COMP \\
    \fa \fh x > 0 & H \\
    \fa \fa 4 = 4 & = \\
    \fa \fa 4 = 2^2 & 5, ARIT \\
    \fa \fa 4 = (1 + 1)^2 & 6, ARIT \\
    \fa \fa x \geq 1 & 4, ARIT \\
    \fa \fa 1 \leq x & 8, ARIT \\
    \fa \fa 4 = (1 + 1)^2 \land 1 \leq x & 7,9,$\land$I \\
    \fa x > 0 \rightarrow 4 = (1 + 1)^2 \land 1 \leq x & 4-10, $\rightarrow$I \\
    \fa (\!|\ x > 0\ |\!)\ i := 1;\ j := 4\ (\!|\ INV\ |\!) & 3,11, PreStren \\
  \end{fitch}
\end{equation*}

Com o \emph{while}, que foi concluído na primeira parte, e a composição, que foi
concluída na seguinda parte, podemos seguir a prova desta maneira:

\begin{equation*}
  \begin{fitch}
    \fb
      \begin{split}
        &(\!|\ INV\ |\!) \\
        &\text{while}\ i \neq x \left\{j := j + 2 \times i + 3;\ i := i + 1\right\} \\
        &(\!|\ INV \land \neg (i \neq x)\ |\!) \\
      \end{split}
      & H \\
    \fj (\!|\ x > 0\ |\!)\ i := 1;\ j := 4\ (\!|\ INV\ |\!) & H \\
    \fa 
      \begin{split}
        &(\!|\ x > 0\ |\!) \\
        &i := 1;\ j := 4\ \text{while}\ i \neq x \left\{j := j + 2 \times i + 3;\ i := i + 1\right\} \\
        &(\!|\ INV \land \neg (i \neq x)\ |\!) \\
      \end{split}
      & 2,1, COMP \\
    \fa \fh (j = (i + 1)^2 \land i \leq x) \land \neg (i \neq x) & H \\
    \fa \fa j = (i + 1)^2 \land i \leq x & 4, $\land$EL \\
    \fa \fa j = (i + 1)^2 & 5, $\land$EL \\
    \fa \fa \neg (i \neq x) & 4, $\land$ER \\
    \fa \fa i = x & 7, LÓGICA \\
    \fa \fa j = (x + 1)^2 & =E$\left\{6,8\right\}$ \\
    \fa INV \land \neg (i \neq x) \rightarrow j = (x + 1)^2 & 4-9, $\rightarrow$I \\
    \fa
      \begin{split}
        &(\!|\ x > 0\ |\!) \\
        &i := 1;\ j := 4\ \text{while}\ i \neq x \left\{j := j + 2 \times i + 3;\ i := i + 1\right\} \\
        &(\!|\ j = (x + 1)^2\ |\!) \\
      \end{split}
      & 3,10,PosWeak \\
  \end{fitch}
\end{equation*}


\section{Algoritmo 2}
\label{sec:algo2}

O algotimo 2 retorna o n-ésimo termo de uma progressão geométrica de razão
\emph{y} e como termo inicial o valor \emph{x}. Este algoritmo pode ser
expresso, utilizando a notação de Hoare, da seguinte forma:

$$(\!|\ x\ \geq\ 0\ \wedge x\ =\ x_0\ \wedge\ y=y_0|\!) i\ :=\ 0; j :=\ x;\
while\ i \neq\ n \{j\ :=\ j\ *\ y;\ i\ :=\ i\ +\ 1\}(\!| j=x y^n |\!) $$


\bibliographystyle{sbc}
\bibliography{report}

\end{document}
